\documentclass[11pt]{article}
\usepackage{amssymb}
\usepackage[english]{babel}
\usepackage{fullpage}
\usepackage{graphicx,multirow}
\usepackage{caption}
\captionsetup{font=bf,belowskip=8pt}
\usepackage{hyperref}
\usepackage{amsmath}
\usepackage{enumitem}
\usepackage{subfig}
\usepackage{placeins}

\begin{document}
\begin{titlepage} 
	\newcommand{\HRule}{\rule{\linewidth}{0.5mm}}
	\center
	\textsc{\LARGE Polytechnique Montréal}\\[1.5cm]
	\textsc{\Large LOG8415 : Lab 1}\\[0.5cm]
	\textsc{\large Advanced Concepts in Cloud Computing}\\[0.5cm]
	\HRule\\[0.4cm]
	{\huge\bfseries Selecting VM instances in the Cloud through
	benchmarking}\\[0.4cm]
	\HRule\\[1.5cm]
	{\large\textit{Authors}}\\
	Anis \textsc{Zouatene} (1963304)\\
	Aleksandar \textsc{Stijelja} ( )\\
	Amin \textsc{} ()\\
    Reza \textsc{} ()\\
	\vfill\vfill\vfill {\large\today} \vfill\vfill
	\includegraphics{poly_logo.png}\\[1cm]
	\vfill
\end{titlepage}

\tableofcontents

\section{Abstract}
	\paragraph{} Not all instances of virtual machines are the same. 
    They provide similar usability but have different capabilities 
    and characteristics. As such, even though two instances look similar, 
    they will not provide the same performance. Instances are then divided 
    into multiple categories to best suit what the user wants for his usage. 
    To really make sure an instance is right for us, we would need to put her 
    into a benchmark test. In this lab, we’ll do exactly that for 2 types of 
    instances that will be available on AWS EC2. AmazonWeb Services (AWS) is 
    a leading infrastructure as a service Cloud provider. One of their products, 
    the Amazon Elastic Compute Cloud (Amazon EC2) is a web-based service that 
    allows customers to run application programs in the Amazon Web Services (AWS) 
    public cloud. With EC2, we will be able to create those virtual machines.

	\paragraph{Keywords:}AWS, AWS EC2, Benchmark, VM Performance, Cloud Application
	\pagebreak



\end{document}