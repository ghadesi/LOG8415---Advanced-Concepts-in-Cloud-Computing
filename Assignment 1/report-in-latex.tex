\documentclass[12pt]{article}
\usepackage{amssymb}
\usepackage[english]{babel}
\usepackage{fullpage}
\usepackage{graphicx,multirow}
\usepackage{caption}
\captionsetup{font=bf,belowskip=8pt}
\usepackage{hyperref}
\usepackage{amsmath}
\usepackage{enumitem}
\usepackage{subfig}
\usepackage{placeins}


\begin{document}
\begin{titlepage} 
	\newcommand{\HRule}{\rule{\linewidth}{0.5mm}}
	\center
	\textsc{\LARGE Polytechnique Montréal}\\[1.5cm]
	\textsc{\Large LOG8415 : Lab 1}\\[0.5cm]
	\textsc{\large Advanced Concepts in Cloud Computing}\\[0.5cm]
	\HRule\\[0.4cm]
	{\huge\bfseries Selecting VM instances in the Cloud through
	benchmarking}\\[0.4cm]
	\HRule\\[1.5cm]
	{\large\textit{Authors}}\\
	Anis \textsc{Zouatene} (1963304)\\
	Aleksandar \textsc{Stijelja} ( )\\
	Amin \textsc{} ()\\
    Reza \textsc{} ()\\
	\vfill\vfill\vfill {\large\today} \vfill\vfill
	\includegraphics{poly_logo.png}\\[1cm]
	\vfill
\end{titlepage}


\section{Abstract}
	\paragraph{} Not all instances of virtual machines are the same. 
    They provide similar usability but have different capabilities 
    and characteristics. As such, even though two instances look similar, 
    they will not provide the same performance. Instances are then divided 
    into multiple categories to best suit what the user wants for his usage. 
    To really make sure an instance is right for us, we would need to put her 
    into a benchmark test. In this lab, we’ll do exactly that for 2 types of 
    instances that will be available on AWS EC2. AmazonWeb Services (AWS) is 
    a leading infrastructure as a service Cloud provider. One of their products, 
    the Amazon Elastic Compute Cloud (Amazon EC2) is a web-based service that 
    allows customers to run application programs in the Amazon Web Services (AWS) 
    public cloud. With EC2, we will be able to create those virtual machines.

	\paragraph{Keywords:}AWS, AWS EC2, Benchmark, VM Performance, Cloud Application
	\pagebreak

\section{Introduction} \label{sec:introduction}
	\paragraph{} As said previously, in this lab we will be focusing the majority of
	it in the Amazon Elastic Compute Cloud (Amazon EC2) to create our virtual machines.
	Since a lot of instances meet similar requirements, finding the right type might be
	challenging to a user. The solution to that would be to try all the instances we want 
	to use, benchmark them by making them go through a load and finally we get the results 
	to make our final decision on which one to take.
	\cite{1}\bigskip

	Thus, the goal of this lab is to benchmark 2 different types of EC2 instances, being 
	M4.large and T2.large. The total number of EC2 instances would be 9 in total and on top 
	of it we will have to create 2 clusters in the target groups that will both contain one 
	specific type of the 2 said previously. Of course, for each of these 2 clusters we would 
	need an Application Load Balancer (ALB). Thereafter, in each instance we will need to deploy 
	a Flask application, do our tests/benchmark then report the results.
	\bigskip

	In this paper, we talk about our methodology which will include virtual machines, the elastic 
	load balancer, Flask and how we will conduct our analysis on our VM.
	\bigskip

	\pagebreak


\section{Methodology} \label{sec:methodology}
	\paragraph{} In this section, we will go over the different sections that we 
	worked on to be able to benchmark properly our 9 different instances. We will first
	go over our virtual machines (instances), our  Elastic load balancer, Flask and 
	our benchmark analysis.

	\subsection{Virtual Machines}
		\paragraph{} 
		\cite{2}\bigskip

\end{document}