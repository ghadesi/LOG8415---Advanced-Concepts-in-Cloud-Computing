\documentclass[12pt]{article}
\usepackage{amssymb}
\usepackage[english]{babel}
\usepackage{fullpage}
\usepackage{graphicx,multirow}
\usepackage{caption}
\usepackage[margin=1.52in]{geometry}
\captionsetup{font=bf,belowskip=11pt}
\usepackage{hyperref}
\usepackage{amsmath}
\usepackage{enumitem}
\usepackage{subfig}
\usepackage{placeins}
\usepackage{tabularray}


\begin{document}
\begin{titlepage} 
	\newcommand{\HRule}{\rule{\linewidth}{0.5mm}}
	\center
	\textsc{\LARGE Polytechnique Montréal}\\[1.5cm]
	\textsc{\Large LOG8415 : Lab 2}\\[0.5cm]
	\textsc{\large Advanced Concepts in Cloud Computing}\\[0.5cm]
	\HRule\\[0.4cm]
	{\huge\bfseries MapReduce with Hadoop on AWS (or Azure)}\\[0.4cm]
	\HRule\\[1.5cm]
	{\large\textit{Authors}}\\
	Anis \textsc{Zouatene} (1963304)\\
	Aleksandar \textsc{Stijelja} (1959772)\\
	Amin \textsc{Ghadesi} (2121658)\\
    Reza \textsc{Rouhghalandari} (2153395)\\
	\vfill\vfill\vfill {\large\today} \vfill\vfill
	\includegraphics{poly-logo.png}\\[1cm]
	\vfill
\end{titlepage}


\section{Abstract}
	\paragraph{} The programming model known as "MapReduce" facilitates concurrent processing by splitting 
	petabytes of data into smaller chunks, and processing them in parallel on Hadoop commodity servers. 
	In the end, it aggregates all the data from multiple servers to return a consolidated output back to 
	the application.
		
	\paragraph{} However we have different ways to manage Big Data sets, such as Apache Hadoop or Apache 
	Spark. In this paper, we will explore both softwares and compare their differences and evaluate their 
	performances by conducting a few experiments.

	\paragraph{Keywords:}Big Data, Azure, MapReduce, Spark, Hadoop, Big Data.
	\pagebreak

\section{Introduction} \label{sec:introduction}
	\paragraph{} As 
	\cite{1}\bigskip

    \noindent Thus, the goal of this lab is to benchmark 2 different types of EC2 instances, being M4.large and T2.large. The total number of EC2 instances would be 9 in total and on top of it we will have to create 2 clusters in the target groups that will both contain one specific type of the 2 said previously. Of course, for each of these 2 clusters we would need an Application Load Balancer (ALB). Thereafter, in each instance we will need to deploy a Flask application, do our tests/benchmark then report the results.
	\bigskip

    \noindent In this paper, we talk about our methodology which will include virtual machines, the elastic 
	load balancer, Flask and how we will conduct our analysis on our VM.
	\bigskip

	\pagebreak




\end{document}